\documentclass{article}

\usepackage[margin=.75in]{geometry}
\usepackage{amsmath}
\usepackage{amssymb}
\usepackage[shortlabels]{enumitem}
\usepackage{pgfplots}
\usepackage{circuitikz}
\usepackage{float}

\author{Damien Prieur}
\title{Homework 3 \\ ECES 512}
\date{}

\begin{document}

\maketitle

\section*{Problem 1}
Given the system
$\frac{dx}{dt} =
\begin{bmatrix}
0 &  1 & 2 \\
0 & -2 & 3 \\
0 &  0 & 0 \\
\end{bmatrix}
x +
\begin{bmatrix}
1 \\
1 \\
0 \\
\end{bmatrix}
u
$
and the vector output $ y = \begin{bmatrix} 0 & 1 & 0 \end{bmatrix}x + 2 u $
\begin{enumerate}[a.]
\item What are the eigenvalues of matrix $A$?
\newline
$$ det(\lambda I - A) = 0 \qquad
det(
\begin{bmatrix}
\lambda &  -1 & -2 \\
0 & \lambda+2 & -3 \\
0 &  0 & \lambda \\
\end{bmatrix}
)
=0
$$
$$\lambda(\lambda+2)(\lambda)= \lambda^2(\lambda-2)=0 $$
$$\lambda_1 = 0 \quad \lambda_2 = 0 \quad \lambda_3 = -2 $$


\item Compute the transfer function $G(s)$.
\newline
$$ G(s) = C(sI-A)^{-1}B + D $$
$$
G(s) =
\begin{bmatrix} 0 & 1 & 0 \end{bmatrix}
(
\begin{bmatrix}
s & 0 & 0 \\
0 & s & 0 \\
0 & 0 & s \\
\end{bmatrix}
-
\begin{bmatrix}
0 &  1 & 2 \\
0 & -2 & 3 \\
0 &  0 & 0 \\
\end{bmatrix}
)
^{-1}
\begin{bmatrix}
1 \\
1 \\
0 \\
\end{bmatrix}
+ 2
$$
$$
G(s) =
\begin{bmatrix} 0 & 1 & 0 \end{bmatrix}
(
\begin{bmatrix}
s &  -1 & -2 \\
0 & s+2 & -3 \\
0 &  0  &  s \\
\end{bmatrix}
)
^{-1}
\begin{bmatrix}
1 \\
1 \\
0 \\
\end{bmatrix}
+ 2
$$
We can use the formula $A^{-1} = \frac{1}{det(A)} adj(A)$
$$
\frac{1}{s^2(s+2)}
\begin{bmatrix}
s(s+2) & 0   & 0 \\
s      & s^2 & 0 \\
2s+7   & 3s  & s(s+2) \\
\end{bmatrix}
^T
=
\frac{1}{s^2(s+2)}
\begin{bmatrix}
s(s+2) & s   & 2s+7 \\
0      & s^2 & 3s \\
0      & 0   & s(s+2) \\
\end{bmatrix}
$$
Plugging this in we can find $G(s)$
$$
G(s) =
\frac{1}{s^2(s+2)}
\begin{bmatrix} 0 & 1 & 0 \end{bmatrix}
\begin{bmatrix}
s(s+2) & s   & 2s+7 \\
0      & s^2 & 3s \\
0      & 0   & s(s+2) \\
\end{bmatrix}
\begin{bmatrix}
1 \\
1 \\
0 \\
\end{bmatrix}
+ 2
$$
$$
G(s) =
\frac{1}{s^2(s+2)}
\begin{bmatrix} 0 & s^2 & 3s \end{bmatrix}
\begin{bmatrix}
1 \\
1 \\
0 \\
\end{bmatrix}
+ 2
=
\frac{s^2}{s^2(s+2)} + 2
=
\frac{1}{(s+2)} + 2
$$
$$ G(s) = \frac{2s+5}{s+2} $$


\item Find the pole(s) of the transfer function.
\newline
The only pole that isn't cancelled is $s=-2$.

\item Explain the difference you see between the eigenvalues of $A$ and the poles of the system.
\newline
If we look at the transfer function before we cancelled some terms $G(s) = \frac{s^2}{s^2(s+2)} + 2 $.
We can see that all of the eigenvalues are poles of this equation but the pole $s=0$ gets cancelled by the zero in the numerator.

\item Is the system BIBO stable? Why?
\newline
This system is BIBO stable as the only pole that weren't cancelled are in the left half plane ($s=-2$).

\end{enumerate}


\newpage
\section*{Problem 2}
If the output of the system in problem 1 is changed to $ y = \begin{bmatrix} 1 & .5 & 0 \end{bmatrix}x$
\begin{enumerate}[a.]
\item What is the new transfer function?
\newline
$$ G(s) = C(sI-A)^{-1}B + D $$
$$
G(s) =
\begin{bmatrix} 1 & .5 & 0 \end{bmatrix}
(
\begin{bmatrix}
s & 0 & 0 \\
0 & s & 0 \\
0 & 0 & s \\
\end{bmatrix}
-
\begin{bmatrix}
0 &  1 & 2 \\
0 & -2 & 3 \\
0 &  0 & 0 \\
\end{bmatrix}
)
^{-1}
\begin{bmatrix}
1 \\
1 \\
0 \\
\end{bmatrix}
+ 2
$$
The inner term is identical so we can just replace it then perform the required matrix multiplications.
$$
G(s) =
\frac{1}{s^2(s+2)}
\begin{bmatrix} 1 & .5 & 0 \end{bmatrix}
\begin{bmatrix}
s(s+2) & s   & 2s+7 \\
0      & s^2 & 3s \\
0      & 0   & s(s+2) \\
\end{bmatrix}
\begin{bmatrix}
1 \\
1 \\
0 \\
\end{bmatrix}
+ 2
$$
$$
G(s) =
\frac{1}{s^2(s+2)}
\begin{bmatrix} 1 & .5 & 0 \end{bmatrix}
\begin{bmatrix}
s(s+2) + s  \\
s^2         \\
0           \\
\end{bmatrix}
+ 2
$$
$$
G(s) =
\frac{s(s+2) + s  + .5 s^2}{s^2(s+2)}
+ 2
$$
$$
G(s) =
\frac{1.5s(s+ 2)}{s^2(s+2)}
+ 2
$$
$$ G(s) = \frac{1.5}{s} + 2 =  \frac{1.5 + 2s}{s}$$

\item Find the poles.
\newline
The new pole is located at $s=0$

\item Is the system with the new output still BIBO stable? Why?
\newline
The new system is not BIBO stable as the pole is not in the left half plane $Re(s) < 0$

\item Compare the BIBO stablility of the systems from both problems. Explain why they are identical/different.
\newline
The second system is not stable while the first is due to the introduction of the first state variable in the output.
This state is probably not internally stable but since we are only looking at the bounded input and output the internal state doesn't matter to us.
By including this internal unstable state the system's output is no longer finite.
We can see this mathematically as the new transfer function's poles are no longer located in the left half plane.

\end{enumerate}

\end{document}
