\documentclass{article}

\usepackage[margin=.75in]{geometry}
\usepackage{amsmath}
\usepackage{amssymb}
\usepackage[shortlabels]{enumitem}
\usepackage{pgfplots}
\usepackage{circuitikz}
\usepackage{float}

\author{Damien Prieur}
\title{Homework 2 \\ ECES 512}
\date{}

\begin{document}

\maketitle

\section*{Problem 1}
Given the time varying system below:
$$
\dot{x} =
\begin{bmatrix}
-1 & e^{2t} \\
0 & -1 \\
\end{bmatrix}
x
$$
\begin{enumerate}[a.]
\item Find the fundamental matrix of the system.
\newline
First we can expand the state equations into two state equations.
$$ \dot{x_1} = -x_1 + e^{2t}x_2 \qquad \dot{x_2} = -x_2 $$
We can start by solving the second equation trivially.
$$ x_2(t) = c_2e^{-t} $$
Substituting the solution into the first equation we get
$$ \frac{dx_1}{dt} = -x_1 + e^{2t} c_2e^{-t} $$
We can solve this equation using an integrating factor.
$$\frac{dy}{dx} + p(x)y = q(x) \implies y = \frac{\int \mu q(x) dx}{\mu} \quad \text{Where} \quad \mu = e^{\int p(x) dx}$$
Rewriting the first equation we get
$$ \frac{x_1}{dt} + 1 x_1 = c_2e^{t} \implies p(x) = 1 \quad q(x) = c_2e^{t} $$
$$ \mu = e^{\int 1 dt}  = e^t $$
$$ x_1(t) = \frac{\int e^t c_2e^{t}dt}{e^t} = e^{-t} \int c_2e^{2t} dt$$
$$ x_1(t) = e^{-t}(\frac{c_2}{2}e^{2t} + c_1) $$
$$ x_1(t) = c_1e^{-t}+\frac{c_2}{2}e^{t} $$
Now we can look at 2 initial state solutions. For simplicity $t_0 = 0$, $x^{(1)}(t_0) = \begin{bmatrix} 1 \\ 0 \end{bmatrix}$, $x^{(1)}(t_0) = \begin{bmatrix} 0 \\ 1 \end{bmatrix}$
$$x^{(1)}(0) = \begin{bmatrix} 1 \\ 0 \end{bmatrix} \implies
x^{(1)}(0) =
\begin{bmatrix}
e^{-t}\\\
0\\
\end{bmatrix}
$$
$$x^{(2)}(0) = \begin{bmatrix} 0 \\ 1 \end{bmatrix} \implies
x^{(2)}(0) =
\begin{bmatrix}
\frac{1}{2} e^{t}\\
e^{-t}\\
\end{bmatrix}
$$
The two initial conditions are trivially independent as well as it is the diagonal matrix.
The fundamental matrix is the two particular solutions concatenated.
$$
X(t) =
\begin{bmatrix}
e^{-t} & \frac{1}{2} e^{t} \\
0 & e^{-t} \\
\end{bmatrix}
$$
This matrix is also independent by looking at the determinant
$$ det(X) = e^{-2t} $$

\item Find the state transition matrix for the system.
\newline
The state transition matrix can be found using $ \Phi (t,t_0) = X(t)X^{-1}(t_0)$.
First we compute $X^{-1}(t)$, since it's a 2x2 matrix it's trivial
$$X^{-1}(t) =
\frac{1}{e^{-2t}}
\begin{bmatrix}
e^{-t} & -\frac{1}{2} e^{t} \\
0 & e^{-t} \\
\end{bmatrix}
=
\begin{bmatrix}
e^{t} & -\frac{1}{2}e^{3t} \\
0 & e^{t} \\
\end{bmatrix}
$$
$$
\Phi(t,t_0) =
\begin{bmatrix}
e^{-t} & \frac{1}{2} e^{t} \\
0 & e^{-t} \\
\end{bmatrix}
\begin{bmatrix}
e^{t} & -\frac{1}{2}e^{3t} \\
0 & e^{t} \\
\end{bmatrix}
=
\begin{bmatrix}
e^{t - t_0} & \frac{1}{2}e^{t+t_0}-\frac{1}{2} e^{-(t - 3 t_0)} \\
0 & e^{t-t_0} \\
\end{bmatrix}
$$

\item What is the time domain solution if the system is unforced with initial condition $x(0) = \begin{bmatrix}1 \\ 2 \end{bmatrix}$?
\newline
The solution of a Linear Time Varying System is given by
$$x(t) = \Phi(t,t_0)x_0 + \int_{t_0}^{t}\Phi(t,\tau)B(\tau)u(\tau)d\tau $$
Since we are looking at the Zero input response we can just use the first term.
$$x(t) =
\begin{bmatrix}
e^{t - t_0} & \frac{1}{2}e^{t+t_0}-\frac{1}{2} e^{-(t - 3 t_0)} \\
0 & e^{t-t_0} \\
\end{bmatrix}
\begin{bmatrix}
1 \\
2
\end{bmatrix}
=
\begin{bmatrix}
e^{t - t_0} + e^{t+t_0} - e^{-(t - 3 t_0)} \\
2 e^{t-t_0} \\
\end{bmatrix}
$$

\item With the input $u(t)$ and the same initial condition in part c, what is the general time domain solution? (You don't need to expand the integral.)
\newline
The solution of a Linear Time Varying System is given by
$$x(t) = \Phi(t,t_0)x_0 + \int_{t_0}^{t}\Phi(t,\tau)B(\tau)u(\tau)d\tau $$
The first term will be the same
$$x(t) =
\begin{bmatrix}
e^{t - t_0} + e^{t+t_0} - e^{-(t - 3 t_0)} \\
2 e^{t-t_0} \\
\end{bmatrix}
+
\int_{0}^{t}
\begin{bmatrix}
e^{t - t_0} & \frac{1}{2}e^{t+t_0}-\frac{1}{2} e^{-(t - 3 t_0)} \\
0 & e^{t-t_0} \\
\end{bmatrix}
B(\tau)
u(\tau)
d\tau
$$


\end{enumerate}

\newpage
\section*{Problem 2}
Now with the state space of your own project:
\begin{enumerate}[a.]
\item What is the input and output?
\newline
The input is a Voltage.
The output is the distance the ball is away from the coil.

\item What are the states and their initial conditions?
\newline
The states are the current, distance from the coil and velocity of the ball relative to the coil.
$$
\begin{bmatrix}
x_1 \\
x_2 \\
x_3 \\
\end{bmatrix}
=
\begin{bmatrix}
\text{current} \\
\text{position} \\
\text{velocity} \\
\end{bmatrix}
$$
The initial conditions will be based on the point we linearize around.
The current and velocity will start at the equilibrium point while the position will be near the equilibrium point but my be above or below by some margin.
$$
x_0
=
\begin{bmatrix}
7   \\
.01 \\
0   \\
\end{bmatrix}
$$
The equilibrium point is derived below in part C

\item What is the time domain solution of your system?
\newline
Here is the full system
$$
\begin{bmatrix}
\dot{x}_1 \\
\dot{x}_2 \\
\dot{x}_3
\end{bmatrix}
=
\begin{bmatrix}
\frac{u(t)-Rx_1(t)}{L} \\
x_3(t) \\
-\frac{Kx_1^2(t)}{mx_2(t)} + g
\end{bmatrix}
$$
$$
y(t) = x_2(t)
$$
First we must linearize around an equilibrium point.
$$
\begin{bmatrix}
\dot{x}_1 \\
\dot{x}_2 \\
\dot{x}_3
\end{bmatrix}
=
\begin{bmatrix}
0 \\
0 \\
0 \\
\end{bmatrix}
=
\begin{bmatrix}
\frac{u(t)-Rx_1(t)}{L} \\
x_3(t) \\
-\frac{Kx_1^2(t)}{mx_2(t)} + g
\end{bmatrix}
$$
Which gives us
$$ x_1(t) = \frac{u(t)}{R}, \qquad x_3(t) = 0, \qquad x_2(t) = \frac{Ku^2(t)}{mgR^2} $$
\begin{minipage}{.45\textwidth}
    For a constant input voltage we get
    $$ y_0 = \frac{Kv_0^2}{mgR^2} $$
\end{minipage}
\hfill
\begin{minipage}{.45\textwidth}
    If we want to hold the ball at a height $y_0$
    $$ v_0 = \sqrt{\frac{y_0mgR^2}{K}} $$
\end{minipage}%
Now the general form for the linearized solution
$$
\begin{bmatrix}
\dot{x}_1 \\
\dot{x}_2 \\
\dot{x}_3
\end{bmatrix}
=
\begin{bmatrix}
\frac{u(t)-Rx_1(t)}{L} \\
x_3(t) \\
-\frac{Kx_1^2(t)}{mx_2(t)} + g
\end{bmatrix}
$$
$$
A
=
\frac{\partial h}{\partial x}
=
\begin{bmatrix}
-\frac{R}{L} & 0 & 0\\
0 & 0 & 1\\
-\frac{2Kx_1(t)}{mx_2(t)} & \frac{K}{m}(\frac{x_1(t)}{x_2(t)})^2 & 0\\
\end{bmatrix}
=
\begin{bmatrix}
-\frac{R}{L} & 0 & 0\\
0 & 0 & 1\\
-\frac{2Kx_{01}}{mx_{02}} & \frac{K}{m}(\frac{x_{01}}{x_{02}})^2 & 0\\
\end{bmatrix}
$$
$$
B
=
\frac{\partial h}{\partial u}
=
\begin{bmatrix}
\frac{1}{L} \\
0 \\
0 \\
\end{bmatrix}
\qquad
C
=
\begin{bmatrix}
0 & 1 & 0
\end{bmatrix}
\qquad
D
=
\begin{bmatrix}
0
\end{bmatrix}
$$
Now if we linearize around the input voltage $= 7V$
$$
v_0 = 7 \qquad
x_0
=
\begin{bmatrix}
7 \\
.00998 \\
0
\end{bmatrix}
\qquad
A
=
\begin{bmatrix}
-100 & 0 & 0 \\
0 & 0 & 1 \\
-2.803 & 982 & 0
\end{bmatrix}
\qquad
B
=
\begin{bmatrix}
100 \\
0 \\
0
\end{bmatrix}
$$
Now we find the time domain solution of the linearized system
$$
\mathbf{x}(t)
=
e^{\mathbf{At}}\mathbf{x}(0) + \int_0^t e^{\mathbf{A}(t-\tau)}\mathbf{B}\mathbf{u}(\tau) d\tau
$$
First we need to find the matrix exponential
\newline
Eigenvalues
$$
(\lambda I - A)x = 0
\quad
\begin{bmatrix}
\lambda + 100 & 0 & 0 \\
0 & \lambda & 1 \\
2.803 & -982 & \lambda
\end{bmatrix}
x = 0
$$
$$
98200 + 982 \lambda - 100 \lambda^2 - \lambda^3
$$
$$
\implies \lambda_1 = -100 \lambda_2 = 31.3369 \quad \lambda_3 = -31.3369
$$
Jordan Decomposition
$$
(A-\lambda_1 I)q_1=0,
\begin{bmatrix}
0 & 0 & 0 \\
0 & 100 & 1 \\
-2.803 & 982 & 100
\end{bmatrix}
q_1
=
0
\implies
q_1 = \begin{bmatrix} 3217.267 \\ 1 \\ -100 \end{bmatrix}
$$
$$
\begin{bmatrix}
-131.3369 & 0 & 0 \\
0 & -31.3369 & 1 \\
-2.803 & 982 &-31.3369 
\end{bmatrix}
q_2
=0
\implies
q_2 = \begin{bmatrix} 0 \\ 1 \\ 31.3369 \end{bmatrix}
$$
$$
\begin{bmatrix}
-68.6631 & 0 & 0 \\
0 & 31.3369 & 1 \\
-2.803 & 982 & 31.3369
\end{bmatrix}
q_3
=
0
\implies
q_3 = \begin{bmatrix} 0 \\ 1 \\ -31.3369 \end{bmatrix}
$$
Normalizing eigenvectors
$$
q_1=
\begin{bmatrix} .999517 \\ -3.1067\times10^{-4} \\ .031067 \end{bmatrix}
\quad
q_2=
\begin{bmatrix} 0 \\ .031895 \\ .999491 \end{bmatrix}
q_3=
\begin{bmatrix} 0 \\ .031895 \\ -.999491 \end{bmatrix}
$$
Jordan Normal Form
$$
Q
=
\begin{bmatrix}
.999517             & 0 & 0 \\
-3.1067\times10^{-4} & .031895 & .031895 \\
.031067            & .999491 & -.999491
\end{bmatrix}
$$
$$
J
=
\begin{bmatrix}
-100 & 0 & 0 \\
0 & 31.3369 & 0 \\
0 & 0 & -31.3369
\end{bmatrix}
$$
$$
Q^{-1}
=
\begin{bmatrix}
1.00048 & 0. & 0. \\
-0.0106764 & 15.6764 & 0.500255 \\
 0.0204215 & 15.6764 & -0.500255 \\
\end{bmatrix}
$$
Matrix Exponential
$$ e^{At} = Qe^JQ^{-1}
=
Q
\begin{bmatrix}
e^{-100t} & 0 & 0 \\
0 & e^{31.3369t} & 0 \\
0 & 0 & e^{-31.3369t}
\end{bmatrix}
Q^{-1}
$$
This Expression becomes very large so each column of the matrix will be broken up with the first column followed by the next two.
$$
\begin{bmatrix}
e^{\lambda_1 t} 																								\\
-3.10823\times 10^{-4} e^{\lambda_1 t}+6.51349\times 10^{-4} e^{\lambda_3 t}-3.40526\times 10^{-4} e^{\lambda_2 t} \\
 0.0310823 e^{\lambda_1 t}-0.0204112 e^{\lambda_3 t}-0.010671 e^{\lambda_2 t} 										\\
\end{bmatrix}
$$
$$
\begin{bmatrix}
0 								   			& 0 \\
0.5 e^{\lambda_3 t}+0.5 e^{\lambda_2 t} 			& 0.0159556 e^{\lambda_2 t}-0.0159556 e^{\lambda_3 t} \\
15.6684 e^{\lambda_2 t}-15.6684 e^{\lambda_3 t} 	& 0.5 e^{\lambda_3 t}+0.5 e^{\lambda_2 t} \\
\end{bmatrix}
$$

Solution
$$\mathbf{u}(t) = \begin{bmatrix} 7 \end{bmatrix} $$
$$
\mathbf{x}(t)
=
e^{\mathbf{At}}\mathbf{x}(0) + \int_0^t e^{\mathbf{A}(t-\tau)}\mathbf{B}\mathbf{u}(\tau) d\tau
$$
$$
\mathbf{x}(t)
=
e^{\mathbf{A}t}\mathbf{x}(0)
+
\begin{bmatrix}
7 -7 e^{-100 t} \\
0.0199807 +0.00217576 e^{-100 t}-0.0145498 e^{-31.3369 t}-0.00760664 e^{31.3369 t} \\
-0.217576 e^{-100. t}+0.455944 e^{-31.3369 t}-0.238368 e^{31.3369 t} \\
\end{bmatrix}
$$
$$
x(t) =
\begin{bmatrix}
7 \\
0.0199807\, -0.00499033 e^{-31.3369 t}-0.00499033 e^{31.3369 t} \\
0.156381 e^{-31.3369 t}-0.156381 e^{31.3369 t} \\
\end{bmatrix}
$$

\item Simulate your system and plot the states and output in Matlab.
\newline
\begin{figure}[h!]
\includegraphics[width=1\linewidth]{{images/linearized_7.000000_input_7.000000_initial_pos_-0.010000}.png}
\end{figure}
\newline
The Current appears to be the correct equation which make sense due to the fact that that part of the system can be described by a linear system.
The Position and velocity graphs appear to be close but not accurate to the simulation. I would attribute this to lack of numerical precision since most of the numbers are quite small and being multiplied making them even smaller.
Also while computing my close form solution I removed terms that were very close to zero so that I didn't have to carry them through the equations.
The solution to the nonlinear differential equation is also graphed, but it is difficult to see as the linearized solutions quickly grow in comparison.
The linearized nonlinear terms cause the system to explode when it leaves the small region around it's linearization region.
\end{enumerate}


\end{document}
