\documentclass{article}

\usepackage[margin=.75in]{geometry}
\usepackage{amsmath}
\usepackage{amssymb}
\usepackage[shortlabels]{enumitem}
\usepackage{pgfplots}
\usepackage{circuitikz}
\usepackage{float}

\author{Damien Prieur}
\title{Homework 1 \\ ECES 512}
\date{}

\begin{document}

\maketitle

\section*{Problem 1}
Given the system with state equations below with unit step input $u(t)$ and initial condition $x_1(0) = 2$, $x_2(0)=3$:
\begin{align*}
\dot{x}_1 &= -2x_1 + u \\
\dot{x}_2 &= \phantom{-}x_1 - x_2
\end{align*}

\begin{enumerate}[a.]
\item Find the state space representation of the system.
$$
\begin{bmatrix}
\dot{x}_1 \\
\dot{x}_2 \\
\end{bmatrix}
=
\begin{bmatrix}
-2 & 0 \\
1  & -1 \\
\end{bmatrix}
\begin{bmatrix}
x_1 \\
x_2 \\
\end{bmatrix}
+
\begin{bmatrix}
1 \\
0 \\
\end{bmatrix}
u
$$

\item Find the time domain solution.\\
Solving for the solution using a convolution.
$$ \mathbf{x}(t)
=
e^{\mathbf{A}t}\mathbf{x}(0) + \int_0^t e^{\mathbf{A}(t-\tau)}\mathbf{B}\mathbf{u}(\tau) d\tau
$$
First we find the matrix exponential of $\mathbf{A}$
$$e^{\mathbf{A}t} = 
\begin{bmatrix}
e^{-2 t} & 0 \\
e^{-2 t} (e^t-1) & e^{-t} \\
\end{bmatrix}
$$
Plugging this into our equation we get
$$ \mathbf{x}(t)
=
\begin{bmatrix}
e^{-2 t} & 0 \\
e^{-2 t} (e^t-1) & e^{-t} \\
\end{bmatrix}
\begin{bmatrix}
2\\
3
\end{bmatrix}
+
\int_0^t 
\begin{bmatrix}
e^{2 \tau -2 t} \\
e^{\tau -2 t} (e^t-e^{\tau }) \\
\end{bmatrix}
\begin{bmatrix}
1\\
0
\end{bmatrix}
1
d\tau
$$
$$ \mathbf{x}(t)
=
\begin{bmatrix}
2 e^{-2 t} \\
2 e^{-2 t} (e^t-1) + 3 e^{-t} \\
\end{bmatrix}
+
\begin{bmatrix}
\frac{1}{2}-\frac{1}{2}e^{-2 t} \\
\frac{1}{2} e^{-2 t} (e^t-1)^2 \\
\end{bmatrix}
$$
$$
x(t) =
\begin{bmatrix}
\frac{3}{2}e^{-2 t}+\frac{1}{2} \\
-\frac{3}{2}e^{-2 t}+4 e^{-t}+\frac{1}{2} \\
\end{bmatrix}
$$
\item Try to sketch your solution by hand. \\
\begin{figure} [H]
    \centering
    \includegraphics[width=.75\linewidth]{{P1_C}.png}
\end{figure}
\item Verify your result in part b and c using Matlab(lsim).
\begin{figure} [H]
    \centering
    \includegraphics[width=.75\linewidth]{{images/p1_predicted_vs_simulation}.png}
\end{figure}
\end{enumerate}

\newpage
\section*{Problem 2}
Given the system with transfer function below with unit step input $u(t)$ and zero initial conditions.
$$ G(s) = \frac{3s + 2}{s(s^2+3s+2)} $$

\begin{enumerate}[a.]
\item Apply partial fraction decomposition to $G(s)$.
$$ \frac{3s + 2}{s(s+2)(s+1)} = \frac{A}{s} + \frac{B}{s+1} + \frac{C}{s+2} $$
$$ 3s + 2 = {A}(s^2+3s+2) + B(s^2+2s) + C(s^2+s) $$
$$ A = 1 \qquad A + B + C = 0 \qquad 3A + 2B + C = 3 $$
$$ A = 1 \qquad B + C = -1 \qquad 2B = -C $$
$$ A = 1 \qquad B = 1 \qquad C = -2 $$
$$ \frac{3s + 2}{s(s+2)(s+1)} = \frac{1}{s} + \frac{1}{s+1} + \frac{-2}{s+2} $$

\item Draw a block diagram of the system.
\begin{figure} [H]
    \centering
    \includegraphics[width=.75\linewidth]{{P2_B}.png}
\end{figure}
\item Find a state space representation of the system.\\
Using controllable Canonical Form
$$G(s) = \frac{Y(s)}{U(s)} = \frac{Z(s)(3s + 2)}{Z(s)s(s^2+3s+2)} $$
$$ y = 3\dot{z} + 2z $$
$$ u = \dddot{z} + 3\ddot{z} +2\dot{z} $$
$$ \begin{bmatrix}q_1\\q_2\\q_3\\\end{bmatrix} = \begin{bmatrix}z\\\dot{z}\\\ddot{z}\\\end{bmatrix}$$
$$\dot{\mathbf{q}} =
\begin{bmatrix}
q_2\\
q_3\\
u - 3q_3 - 2 q_2\\
\end{bmatrix}
$$
$$y = 3 q_2 + 2 q_1 $$
$$\dot{\mathbf{q}} =
\begin{bmatrix}
0 & 1 & 0 \\
0 & 0 & 1 \\
0 &-2 &-3 \\
\end{bmatrix} 
\mathbf{q}
+
\begin{bmatrix}
0 \\
0 \\
1 \\
\end{bmatrix} 
u
$$
$$ y = \begin{bmatrix} 2 & 3 & 0 \end{bmatrix} \mathbf{q} $$
\item Find the time domain solution. \\
Finding the solution using the inverse laplace transform.
$$ U(s) = \frac{1}{s} $$
$$ Y(s) = G(s) U(s) $$
$$ Y(s) = \frac{3s+2}{s^2(s^2+3s+2)} $$
$$ Y(s) = \frac{1}{s^2}+\frac{1}{s+2}-\frac{1}{s+1} $$
Take the inverse Laplace of each term since the Laplace transform is linear we get
$$y(t) = e^{-2 t}-e^{-t} + t$$
\item Verify your result using Matlab(lsim).
\begin{figure} [H]
    \centering
    \includegraphics[width=.75\linewidth]{{images/p2_2_predicted_vs_simulation}.png}
\end{figure}
\end{enumerate}

\end{document}
